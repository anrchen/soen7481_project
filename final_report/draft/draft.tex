
\section{Abstract}


Index terms: software test, static analysis, exception stack trace, log analysis

\section{Introduction}

Challenges

\vspace{-0.1cm}
\begin{itemize} \itemsep 0em
 \item RQ 1: How are recent changes correlated with production errors?

 \tbd {summary of findings}

 \item RQ 2: How are test cases and exception stack trace correlated?

 \tbd {summary of findings}
 
 \item RQ 3: Can production failures contribute to better test cases?

 \tbd {summary of findings}
 
\end{itemize}

\section{Methodology}
We studied 65 user-reported bugs from \tbd{reference} Apache Commons Lang, a Java helper package that contains extra utilities for java.lang API. 

Those failures we studied were collected by \tbd{reference} Defects4J. Each bug consists of the fixing commit number and the related tests that failed before the fix. We selected a specific set of issues that contain exception stack trace. The stack trace will help us to pinpoint the incident and thus, evaluate the root cause of the failure. 

The main methodology is break down into three steps:
\vspace{-0.1cm}
\begin{itemize} \itemsep 0em
 \item Exception and source code mapping
 \item Test and source code mapping
 \item Exception and test mapping
\end{itemize}

\phead{Exception and source code mapping}

\phead{Test and source code mapping}

\phead{Exception and test mapping}


Limitations: 

1) Small sample set. Only 65 bugs \tbd{more} In the scale 

2) Representativeness of the seleted bugs. We only studied bugs reported from Apache Commons Lang project. Some issues, such as misconfigurations, are project-related, which might not provide a generic finding for future research. In consequence, we only list common findings that can also be observed in other Java projects. 

We do our best to categorize production errors into findings to better reflect the characteristics of distinct errors. 

\section{General Findings}


\section{Threats to validity}

\section{Conclusion}
